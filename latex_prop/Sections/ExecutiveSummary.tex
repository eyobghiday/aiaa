
%
\par

% The team's objective this year is to design a remote-controlled, multi-purpose, short-takeoff-and-landing (STOL) aircraft. Additionally, mission requirements dictate that the aircraft must also be capable of chartering passengers and towing a banner. The STOL capabilities are such that the aircraft is limited to a 20 ft takeoff distance with the banner as its payload.

This proposal serves to outline the design, manufacturing, and testing processes of the Illinois Institute of Technology (IIT) 2022-23 DBF team. The team's objective is to design and manufacture an electronic warfare aircraft for various surveillance and signal jamming missions. DBF's mission requirements dictate that the aircraft must fit inside an airline checked shipping box with maximum total dimensions of 62 inches. Additionally, the aircraft must complete all three missions and a ground mission as described in Table \ref{tab:mss}. To maximize overall score, the team first conducted sensitivity analyses based on maximum electronic package weight that can be carried for Mission 2 and the length of jamming antenna that can accommodated for Mission 3. The results of this analysis were incorporated into the team’s preliminary design.          

% \par
% %
% \tikzstyle{decision} = [diamond, fill=iitred, text badly centered, node distance=3cm, inner sep=0pt]
% \tikzstyle{block} = [rectangle, fill=iitred, text centered, rounded corners, minimum height=4em]
% \tikzstyle{line} = [draw, -{Latex[width=5pt,length=7pt]}]
% \tikzstyle{cloud} = [ellipse, fill=iitred, node distance=3cm, minimum height=2em]
% \vspace{0.5cm}

% \begin{figure}[H]
%     \centering
%     \begin{tikzpicture}[node distance = 2cm, auto, text=white, inner sep=5pt, align=center]
%         % Place nodes
%         \node [block] (design) {CAD Design\\\& Analysis};
%         \node [block, right of = design, node distance = 3.5cm] (manufacturing) {Manufacturing};
%         \node [block, right of= manufacturing, node distance = 3.5cm] (assembly) {Assembly};
%         \node [block, right of= assembly, node distance = 3.5cm] (testing) {Testing};
%         \node [block, below of= assembly, node distance = 2cm] (feedback) {Design Feedback};
%         \node [block, right of= testing, node distance = 3.5cm] (final) {Final Aircraft};
        
%         % Draw edges
%         \path [line] (design) -- (manufacturing);
%         \path [line] (manufacturing) -- (assembly);
%         \path [line] (assembly) -- (testing);
%         \path [line] (testing) |- (feedback);
%         \path [line] (feedback) -| (design);
%         \path [line] (testing) -- (final);
%     \end{tikzpicture}
%    \caption{Project Plan}
%     \label{fig:manplan}
% \end{figure}
An overview of the team's structure is shown in Section \ref{s:ms}. Following this, Section \ref{s:cda} discusses the approach used to develop a design as well as the individual sub-teams' methods. The manufacturing methods are explained in Section \ref{s:mp}. Finally, the aircraft will go through a testing regime defined in Section \ref{s:tp}.
%


% Everything after this is conceptual design
% \par
% %
% This analysis shows that maximizing payload capacity and flight speed are the two most significantly weighted variables. Thus, a single motor, conventional monoplane, tail-dragger design was chosen as the most viable option. 
% %
% \par
% %
% To begin the preliminary design, the scoring system was analyzed. With mission scores normalized by the highest achieved score, it is desired to aim for the highest mission score possible. Lift, Weight, Payload Capacity, and Drag were the highest weights when considering the configuration of the aircraft. Overall, the Monoplane configuration was selected due to its exceptional Payload Capacity, Stability, Weight, and ease of Manufacturing. The fuselage was determined to be a single fuse due to its superior lift, weight, drag, and ease of manufacturing compared to a twin-fuse. The placement of the wings will be located high due to its stability and ease of manufacturing. The tail will be conventional due to its rigidity, stability, controls, weight, angle of attack performance, and ease of manufacturing. The double motor was selected as it would provide the most thrust and stability.